\documentclass[12pt]{article}

\usepackage{amssymb}
\usepackage{graphicx}
\usepackage{latexsym}
\usepackage{amssymb,amsmath,amstext}
\usepackage{epsfig}
\usepackage{graphics}
\usepackage{subfigure}
\usepackage{euscript}
\usepackage{amsthm}
\numberwithin{equation}{section}
% THEOREMS
\theoremstyle{plain}% 
\newtheorem{theorem}{Theorem}
\numberwithin{theorem}{section}
\newtheorem{proposition}[theorem]{Proposition}
\newtheorem{example}[theorem]{Example}
\newtheorem{lemma}[theorem]{Lemma}
\newtheorem{corollary}[theorem]{Corollary}
\newtheorem{definition}[theorem]{Definition}
\newtheorem{remark}[theorem]{Remark}
\newtheorem{conjecture}[theorem]{Conjecture}

\newcommand{\Z}{\mathbb{Z}}
\newcommand{\R}{\mathbb{R}}

\begin{document}

\title{\bf Tropicalization of classical moduli spaces}

\section{Optimal antibiotic cyclic}

We are considering $14$ antibiotics, labeled
AM, AMC, AMP, CAZ, CEC, CPD, CPR, CRO, CTT, CTX, CXM, FEP, SAM, TZP,
and ZOX. For each of these $14$ antibiotics, we select exactly one TEM 
{\em fitness landscape}.
Such a landscape  is a real $2 {\times} 2 {\times} 2 {\times} 2$ tensor 
${\bf f} = (f_{ijkl})$. The indices $i,j,k,l$ are $0$ or $1$.
We can identify $f$ with a vector whose coordinates
are indexed by $\{0,1\}^4$.

A {\em mutation model} is a function $M : \R^{16} \rightarrow \R^{16 \times 16}$
that assigns a transition matrix to each fitness landscape.
Recall that a {\em transition matrix} has nonnegative entries
and its rows sum to $1$. The rows and columns of
$M(f)$ are labeled by $\{0,1\}^4$, in some order
that is fixed throughout.
We require that our transition matrices respect the adjacency structure
of the $4$-cube, that is, $M(f)_{a,b} = 0$
unless $a$ and $b$ are vectors in $\{0,1\}^4$ that differ in at most one coordinate.
Thus each row of $M(f)$ has at most $5$ non-zero entries.

Two mutation models are described in Section 3.2 of {\tt projectdraft0204.pdf}.
The second model is obtained by simply considering the
directed graph on $\{0,1\}^4$ where $a \rightarrow b$ means
that $f_a < f_b$. The non-zero diagonal entries of $M(f)$ are
  $M(f)_{a,a} = 1$ if $a$ has no outgoing edges, 
  and the non-zero off-diagonal entries are
$M_{a,b} = 1/\hbox{outdegree(a)}$ for every directed edge $a \rightarrow b$.

Let $f_1,f_2,\ldots,f_{14}$ denote our $14$ given fitness landscape,
with derived transition matrices
 $M(f_1)$, $M(f_2),\ldots, M(f_{14})$.
 Let $\mathcal{W}$ denote a finite set of
 words on the alphabet $\{0,1,\ldots,14\}$.
 These words represent the feasible {\em treatment plans} we are considering.
 Every word $w = w_1 w_2 \cdots w_k$ represents a new transition matrix,
 namely the corresponding
  product of $16 {\times} 16$-matrices
 $$ M[w] \,\, = \,\, M(f_{w_1}) \cdot M(f_{w_2}) \cdot \cdots \cdot M(f_{w_k}).$$
Our task is to solve the following discrete optimization problem: \\
{\em
Maximize the entry upper left entry $M[w]_{0000,0000}$
over all words $w$ in $\mathcal{W}$. }

The methodology used to solve this problem will depend
on the choice of $\mathcal{W}$. For instance, it would be natural
to take $\mathcal{W}$ as the set of all words of length exactly $k$,
for some small positive integer $k$. Then
$\mathcal{W}$ has $14^k$ elements. For $k \leq 5$
we can solve our problem by brute-force enumeration, but
for $k \leq 6$ something more clever will be needed. 
At this point, I do not know whether a polynomial-time algorithm
exists. The problem is reminiscent of the MAP inference problem
for Hidden Markov Models, which can be solved efficiently
by a dynamic programming approach known as as the {\em Baum-Welch algorithm}.
Our problem seems to be more difficult. We will need to do some
literature search and talk to some experts to find more efficient algorithms.
But, for starters, let's run the bruce-force computation
for some small sets $\mathcal{W}$ of treatment plans
that make sense from a biomedical perspective.






\end{document}

